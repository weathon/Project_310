\documentclass{article}
\usepackage[utf8]{inputenc}
\usepackage { hyperref } 
\title{COSC310 Project Final documentation}
\author{Wenqi Guo}
\date{December 2022}

\begin{document}

\maketitle

\section{What Changed}
\begin{enumerate}
\item The user will be shown an Organic Chemistry problem if he/she is too happy and the user can also send 
a copy of the question to his/her number.
\item If the user is too sad, some cool animation will be shown to cheer him/her up.
\end{enumerate}
\section{APIs Used}
\subsection{Vonage}
When the user chooses to send a copy of the question to himself/herself, Vonage will do the job. We called the Vonage API including the user's number, our text (which is a link to the image and the correct answer), and our API key.
\subsection{Matter.js}
Matter.js is a library for physics simulation. We used it to create a cake sprinkles effect) We pass an HTML element to it and added some circles and squares. Then we changed some physical properties of the items (such as resistance, position, and elasticity). The matter.js will do the simulation. 
\section{GituUb Repo}
\url{https://github.com/weathon/Project_310/}
\section{Video}
\url{https://api.weasoft.com/imgs/video.mp4}
\section{Raw \LaTeX\; Documentation}
\url{https://github.com/weathon/Project_310/blob/dev/doc.tex}
\section{Solutions for the OChem questions included}
This is nothing related to COSC 310.
\subsection{Quetion 1}
\subsubsection*{A}
This is correct, it is a Grignard reaction
\subsubsection*{B}
This is E2 since it is elimination and it happened in one step.
\subsubsection*{C}
This is E1
\subsubsection*{D}
It is easier to attack the other carbon in the epoxide. The carbon shown in the question has too much steric hindrance.
\subsection{Question 2}
Since A is not radical, it is the most stable.
\end{document}
